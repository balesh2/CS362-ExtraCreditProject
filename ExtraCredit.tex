\documentclass[letterpaper,10pt]{article}

\usepackage{geometry}
\usepackage{hyperref}
\usepackage{glossaries}
\usepackage[pdftex]{graphicx}
\usepackage{tikz}
\usepackage{wrapfig}
\usepackage{listings}
\usepackage{color}
\geometry{textheight=8.5in, textwidth=6in}

\definecolor{dkgreen}{rgb}{0,0.6,0}
\definecolor{gray}{rgb}{0.5,0.5,0.5}
\definecolor{mauve}{rgb}{0.58,0,0.82}

\lstset{frame=tb,
  language=bash,
    aboveskip=3mm,
      belowskip=3mm,
	showstringspaces=false,
	  columns=flexible,
	    basicstyle={\small\ttfamily},
	      numbers=none,
	        numberstyle=\tiny\color{gray},
		  keywordstyle=\color{blue},
		    commentstyle=\color{dkgreen},
		      stringstyle=\color{mauve},
		        breaklines=true,
			  breakatwhitespace=true,
			    tabsize=3
			    }

\title{Extra Credit Project: Testing of Input Generation Tools}
\author{Helena~Bales\\ \\ CS362-001 \\ Winter 2017}

\parindent = 0.0 in
\parskip = 0.1 in

\begin{document}
\maketitle

\clearpage
\tableofcontents
\clearpage

\section{Introduction}
This is my documentation of testing an input generation tool on an Open Source Java project. 
The project that I have selected is NanoXML. I will provide a description of this project and a 
link to the project's webpage below. I will test this project using a test input generation tool 
called jCute. Further evalutation of this tool will also be included in this document. I will 
evaluate the code coverage of these generated tests in order to judge the effectiveness of jCute. 
Finally, I will evaluate jCute by inserting five faults in the NanoXML project and running the 
tests to see if the faults are caught.

\section{Selection of Open Source Java Project}
The project that I have selected to test is NanoXML. The project's source code is available at: 
\begin{lstlisting}
http://nanoxml.sourceforge.net/orig/download.html
\end{lstlisting}

	\subsection{Project Details}
	NanoXML is a "small non-validating parser for Java", according to the NanoXML "About" 
	page. I will be using the NanoXML-Java branch, which is the standard branch that is 
	recommended for most users. It is structured as a selection of classes in a source tree. 
	I will be using the most recent version of NanoXML, version 2.2.1.

	\subsection{Classes to Test}
	I will be testing the following four classes:
	\begin{enumerate}
		\item{public class StdXMLReader}
		\item{class PIReader}
		\item{public class XMLElement}
		\item{public class XMLException}
	\end{enumerate}

\section{Selection of Input Generation Tool}
I have selected to use jCute as my Input Generation Tool to test. jCute is a Concolic Unit Testing
 Engine that automatically generates unit tests for Java programs. I will evaluate this tool by 
generating tests for NanoXML, then evaluating the code coverage of those tests, and inserting 
faults into the code to see if the tests catch them.

\section{Using the Input Generation Tool}
	\subsection{Generated Tests}
		\subsubsection{Class 1}
		\input{}

		\subsubsection{Class 2}
		\input{}

		\subsubsection{Class 3}
		\input{}

		\subsubsection{Class 4}
		\input{}
	\subsection{Evaluation of Selected Input Generation Tool}
		\subsubsection{Provided Functionality}
		\subsubsection{Implementation Details}
		\subsubsection{Effective Functionality}
		\subsubsection{Ineffective Functionality}
		\subsubsection{Tool Use Cases}
		\subsubsection{Suggested Possible Improvements}
	\subsection{Coverage of Generated Test Inputs}
		\subsubsection{Coverage Data}
		\subsubsection{Evaluation of Code Coverage}
\section{Fault Insertion}
	\subsection{Fault 1}
		\subsubsection{Fault Description}
		\textbf{Class Name: } \\
		\textbf{Line Number: } \\
		\textbf{Description of Change: } \\
		\subsubsection{Status of Test Detection}
		\textbf{Test Case: } \\
	\subsection{Fault 2}
		\subsubsection{Fault Description}
		\textbf{Class Name: } \\
		\textbf{Line Number: } \\
		\textbf{Description of Change: } \\
		\subsubsection{Status of Test Detection}
		\textbf{Test Case: } \\
	\subsection{Fault 3}
		\subsubsection{Fault Description}
		\textbf{Class Name: } \\
		\textbf{Line Number: } \\
		\textbf{Description of Change: } \\
		\subsubsection{Status of Test Detection}
		\textbf{Test Case: } \\
	\subsection{Fault 4}
		\subsubsection{Fault Description}
		\textbf{Class Name: } \\
		\textbf{Line Number: } \\
		\textbf{Description of Change: } \\
		\subsubsection{Status of Test Detection}
		\textbf{Test Case: } \\
	\subsection{Fault 5}
		\subsubsection{Fault Description}
		\textbf{Class Name: } \\
		\textbf{Line Number: } \\
		\textbf{Description of Change: } \\
		\subsubsection{Status of Test Detection}
		\textbf{Test Case: } \\

\section{Conclusion}
\end{document}
